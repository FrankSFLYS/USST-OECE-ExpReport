%! TEX-program = xelatex
\documentclass[sub3section,fontset=fandol]{ctexart}
    % sub3section - 启用 paragraph 之后换行,无需在 ctexset 中使用 runin
\usepackage{geometry}
\usepackage{xeCJK}
\usepackage{enumerate}
\usepackage{amsmath}
\usepackage{amsthm}
\usepackage{caption}
\usepackage{graphicx}
\usepackage{subfig}
\usepackage{tikz}
\usetikzlibrary{arrows}
\usepackage[european]{circuitikz}
\usepackage{pdfpages}
\usepackage{tabularx}
\usepackage{listings}
%\usepackage{titlesec}
%\usepackage{titletoc}
\usepackage{layouts}
\usepackage{ifthen}
\usepackage{xcolor}  % colorlet
\usepackage{mdwlist} % enumerate*

\geometry{a4paper,left=3.18cm,right=3.18cm,top=2.54cm,bottom=2.54cm}
%%%%%%%%%%%%%%%% 与 ctexset 冲突,不再使用
%\titleformat{\paragraph}[block]{\normalsize\bfseries}{\theparagraph}{1em}{} % 设置 \paragraph 后换行
%%%%%%%%%%%%%%%%
% \DeclareCaptionType{code}[代码]
\setcounter{tocdepth}{2}    % 目录深度
\setcounter{secnumdepth}{4} % 编号的深度,4 表示到 paragraph 一级,默认为 2
\colorlet{darkred}{red!50!black}
%\usepackage{titlesec}
%\titleformat{\section}{\centering\Large}{实验\,\thesection}{1em}{}
\usepackage[colorlinks,linkcolor=darkred]{hyperref}
\definecolor{mygreen}{rgb}{0,0.6,0}
\definecolor{mygray}{rgb}{0.5,0.5,0.5}
\definecolor{mymauve}{rgb}{0.58,0,0.82}
\lstset{ %
	backgroundcolor=\color{white},   % choose the background color
	basicstyle=\footnotesize\ttfamily,        % size of fonts used for the code
	columns=fullflexible,
	breaklines=true,                 % automatic line breaking only at 
	%whitespace
	captionpos=t,                    % sets the caption-position to bottom
	tabsize=4,
	commentstyle=\color{mygreen},    % comment style
	escapeinside={\%*}{*)},          % if you want to add LaTeX within your code
	keywordstyle=\color{blue},       % keyword style
	stringstyle=\color{mymauve}\ttfamily,     % string literal style
	frame=single,
	frameround=fttt,
	rulesepcolor=\color{red!20!green!20!blue!20},
	% identifierstyle=\color{red},
	language=Verilog,
	keepspaces=true,
}
\ctexset{
    section = {
        name = {实验,},
        number = {\chinese{section}}
    },
    subsection/number = {\chinese{subsection}},
	subsubsection/number = {\arabic{subsubsection}},
    paragraph/number = {\arabic{subsubsection}.\arabic{paragraph}}
}
\renewcommand{\lstlistingname}{清单}
\newcommand{\lstdir}{listing}
\newcommand{\instabledir}{table}
\newcommand{\instikdir}{tikz}
\newcommand{\inspicdir}{figure}
\newcommand{\elementdir}{element}
\newcommand{\inlineformula}{on}
\newcommand{\includetable}{on}
\newcommand{\includefigure}{on}

\newcommand{\instable}[3]{ %
	\begin{table}[!htb] %
		\ifthenelse{\equal{#1}{}}{}{ %
			\caption{#1} %
			\label{#2} %
		} %
		\centering %
		\ifthenelse{\equal{\includetable}{on}}{ %
			\input{\instabledir/#3} %
		}{} %
	\end{table} %
}
\newcommand{\instik}[3]{ %
	\begin{figure}[!htb] %
		\centering %
		\ifthenelse{\equal{\includefigure}{on}}{ %
			\input{\instikdir/#3} %
		}{} %
		\ifthenelse{\equal{#1}{}}{}{%
			\caption{#1} %
			\label{#2} %
		} %
	\end{figure} %
}
\newcommand{\inspic}[4][\linewidth]{ %
	\begin{figure}[!htb] %
		\centering %
		\ifthenelse{\equal{\includefigure}{on}}{ %
			\includegraphics[width=#1]{\inspicdir/#4} %
		}{} %
		\ifthenelse{\equal{#2}{}}{}{ %
			\caption{#2} %
			\label{#3} %
		} %
	\end{figure} %
}
\newcommand{\insfigure}[2][\linewidth]{%
	\ifthenelse{\equal{\includefigure}{on}}{%
		\includegraphics[width=#1]{\inspicdir/#2}  %
	}{}%
}
\newcommand{\e}[2][]{\ifthenelse
	{\equal{\inlineformula}{on}}{ %
		\ifthenelse{\equal{#1}{}}{$ #2 $}{$\mathsf{ #1 } #2 $}}{}}
%\renewcommand{\proofname}{解}
\renewcommand{\qedsymbol}{\textcolor{darkred}{\rule{1ex}{1.5ex}}}

\captionsetup{font={footnotesize}}
\begin{document}

% Uncomment the line below to disable inline formulas
%\renewcommand{\inlineformula}{off}
% Uncomment the line below to disable all tables
%\renewcommand{\includetable}{off}
% Uncomment the line below to disable all figures
\renewcommand{\includefigure}{off}
\newcommand{\ReportHeadSchool}{上海理工大学光电信息与计算机工程学院}
\newcommand{\ReportHeadPrefix}{《集成电路设计》}
\newcommand{\ReportHeadSuffix}{实验报告}
\newcommand{\ReportAuthor}{名字}
\newcommand{\ReportNumber}{学号}
\newcommand{\ReportMajor}{电子科学与技术}
\newcommand{\ReportYear}{2015 级}
\newcommand{\ReportTeacher}{教师}

\begin{titlepage}
\begin{center}
    \textbf{\fontsize{16bp}{\baselineskip}\selectfont 上海理工大学光电信息与计算机工程学院} \\[12ex]
    \textbf{\fontsize{26bp}{\baselineskip}\selectfont \ReportHeadPrefix\ReportHeadSuffix} \\[5ex]
    \includegraphics[width=8cm]{\elementdir/usst.png}  \\[2ex]
    \zihao{4}
    \renewcommand\arraystretch{2.0}
    \begin{tabularx}{8cm}
    {cXcX}
        专\hspace{2\ccwd}业 & \quad\ReportMajor   \\ \cline{2-2}
        姓\hspace{2\ccwd}名 & \quad\ReportAuthor  \\ \cline{2-2}
        学\hspace{2\ccwd}号 & \quad\ReportNumber  \\ \cline{2-2}
        年\hspace{2\ccwd}级 & \quad\ReportYear    \\ \cline{2-2}
        指导教师            & \quad\ReportTeacher \\ \cline{2-2}
        成\hspace{2\ccwd}绩 & :\\
        教师签字            & :\\
    \end{tabularx}
\end{center}
\end{titlepage}

\pagenumbering{Roman}
\setcounter{page}{1}
\tableofcontents
\clearpage
\pagenumbering{arabic}
\setcounter{page}{1}

\section{进入实验的世界}

\par 你好,\LaTeXe。使用 \verb|minted| 引入 C 的例子:

\begin{minted}{C}
int main(int argc, char* argv[]) {
    printf("Hello, LaTeX!");
    return 0;
}
\end{minted}
%\section{C语言基本语法}

\subsection{实验目的}
\par 了解 C语言。

\subsection{实验过程}
\par C语言基本结构见清单 \ref{code:main}。

\begin{code}
    \caption{main.c}
    \label{code:main}
    \begin{minted}{C}
#include <stdio.h>

int main(int argc, char* argv[]) {
    printf("Hello, world!");

    return 0;
}
    \end{minted}
\end{code}

\subsection{实验结果}
\par 成功输出一个心形。
%\input{exp/exp3}
%\input{exp/exp4}
%\input{exp/exp5}
%\input{exp/exp6}
%\input{exp/exp7}

\end{document}
